
% Default to the notebook output style

    


% Inherit from the specified cell style.




    
\documentclass[11pt]{article}

    
    
    \usepackage[T1]{fontenc}
    % Nicer default font (+ math font) than Computer Modern for most use cases
    \usepackage{mathpazo}

    % Basic figure setup, for now with no caption control since it's done
    % automatically by Pandoc (which extracts ![](path) syntax from Markdown).
    \usepackage{graphicx}
    % We will generate all images so they have a width \maxwidth. This means
    % that they will get their normal width if they fit onto the page, but
    % are scaled down if they would overflow the margins.
    \makeatletter
    \def\maxwidth{\ifdim\Gin@nat@width>\linewidth\linewidth
    \else\Gin@nat@width\fi}
    \makeatother
    \let\Oldincludegraphics\includegraphics
    % Set max figure width to be 80% of text width, for now hardcoded.
    \renewcommand{\includegraphics}[1]{\Oldincludegraphics[width=.8\maxwidth]{#1}}
    % Ensure that by default, figures have no caption (until we provide a
    % proper Figure object with a Caption API and a way to capture that
    % in the conversion process - todo).
    \usepackage{caption}
    \DeclareCaptionLabelFormat{nolabel}{}
    \captionsetup{labelformat=nolabel}

    \usepackage{adjustbox} % Used to constrain images to a maximum size 
    \usepackage{xcolor} % Allow colors to be defined
    \usepackage{enumerate} % Needed for markdown enumerations to work
    \usepackage{geometry} % Used to adjust the document margins
    \usepackage{amsmath} % Equations
    \usepackage{amssymb} % Equations
    \usepackage{textcomp} % defines textquotesingle
    % Hack from http://tex.stackexchange.com/a/47451/13684:
    \AtBeginDocument{%
        \def\PYZsq{\textquotesingle}% Upright quotes in Pygmentized code
    }
    \usepackage{upquote} % Upright quotes for verbatim code
    \usepackage{eurosym} % defines \euro
    \usepackage[mathletters]{ucs} % Extended unicode (utf-8) support
    \usepackage[utf8x]{inputenc} % Allow utf-8 characters in the tex document
    \usepackage{fancyvrb} % verbatim replacement that allows latex
    \usepackage{grffile} % extends the file name processing of package graphics 
                         % to support a larger range 
    % The hyperref package gives us a pdf with properly built
    % internal navigation ('pdf bookmarks' for the table of contents,
    % internal cross-reference links, web links for URLs, etc.)
    \usepackage{hyperref}
    \usepackage{longtable} % longtable support required by pandoc >1.10
    \usepackage{booktabs}  % table support for pandoc > 1.12.2
    \usepackage[inline]{enumitem} % IRkernel/repr support (it uses the enumerate* environment)
    \usepackage[normalem]{ulem} % ulem is needed to support strikethroughs (\sout)
                                % normalem makes italics be italics, not underlines
    

    
    
    % Colors for the hyperref package
    \definecolor{urlcolor}{rgb}{0,.145,.698}
    \definecolor{linkcolor}{rgb}{.71,0.21,0.01}
    \definecolor{citecolor}{rgb}{.12,.54,.11}

    % ANSI colors
    \definecolor{ansi-black}{HTML}{3E424D}
    \definecolor{ansi-black-intense}{HTML}{282C36}
    \definecolor{ansi-red}{HTML}{E75C58}
    \definecolor{ansi-red-intense}{HTML}{B22B31}
    \definecolor{ansi-green}{HTML}{00A250}
    \definecolor{ansi-green-intense}{HTML}{007427}
    \definecolor{ansi-yellow}{HTML}{DDB62B}
    \definecolor{ansi-yellow-intense}{HTML}{B27D12}
    \definecolor{ansi-blue}{HTML}{208FFB}
    \definecolor{ansi-blue-intense}{HTML}{0065CA}
    \definecolor{ansi-magenta}{HTML}{D160C4}
    \definecolor{ansi-magenta-intense}{HTML}{A03196}
    \definecolor{ansi-cyan}{HTML}{60C6C8}
    \definecolor{ansi-cyan-intense}{HTML}{258F8F}
    \definecolor{ansi-white}{HTML}{C5C1B4}
    \definecolor{ansi-white-intense}{HTML}{A1A6B2}

    % commands and environments needed by pandoc snippets
    % extracted from the output of `pandoc -s`
    \providecommand{\tightlist}{%
      \setlength{\itemsep}{0pt}\setlength{\parskip}{0pt}}
    \DefineVerbatimEnvironment{Highlighting}{Verbatim}{commandchars=\\\{\}}
    % Add ',fontsize=\small' for more characters per line
    \newenvironment{Shaded}{}{}
    \newcommand{\KeywordTok}[1]{\textcolor[rgb]{0.00,0.44,0.13}{\textbf{{#1}}}}
    \newcommand{\DataTypeTok}[1]{\textcolor[rgb]{0.56,0.13,0.00}{{#1}}}
    \newcommand{\DecValTok}[1]{\textcolor[rgb]{0.25,0.63,0.44}{{#1}}}
    \newcommand{\BaseNTok}[1]{\textcolor[rgb]{0.25,0.63,0.44}{{#1}}}
    \newcommand{\FloatTok}[1]{\textcolor[rgb]{0.25,0.63,0.44}{{#1}}}
    \newcommand{\CharTok}[1]{\textcolor[rgb]{0.25,0.44,0.63}{{#1}}}
    \newcommand{\StringTok}[1]{\textcolor[rgb]{0.25,0.44,0.63}{{#1}}}
    \newcommand{\CommentTok}[1]{\textcolor[rgb]{0.38,0.63,0.69}{\textit{{#1}}}}
    \newcommand{\OtherTok}[1]{\textcolor[rgb]{0.00,0.44,0.13}{{#1}}}
    \newcommand{\AlertTok}[1]{\textcolor[rgb]{1.00,0.00,0.00}{\textbf{{#1}}}}
    \newcommand{\FunctionTok}[1]{\textcolor[rgb]{0.02,0.16,0.49}{{#1}}}
    \newcommand{\RegionMarkerTok}[1]{{#1}}
    \newcommand{\ErrorTok}[1]{\textcolor[rgb]{1.00,0.00,0.00}{\textbf{{#1}}}}
    \newcommand{\NormalTok}[1]{{#1}}
    
    % Additional commands for more recent versions of Pandoc
    \newcommand{\ConstantTok}[1]{\textcolor[rgb]{0.53,0.00,0.00}{{#1}}}
    \newcommand{\SpecialCharTok}[1]{\textcolor[rgb]{0.25,0.44,0.63}{{#1}}}
    \newcommand{\VerbatimStringTok}[1]{\textcolor[rgb]{0.25,0.44,0.63}{{#1}}}
    \newcommand{\SpecialStringTok}[1]{\textcolor[rgb]{0.73,0.40,0.53}{{#1}}}
    \newcommand{\ImportTok}[1]{{#1}}
    \newcommand{\DocumentationTok}[1]{\textcolor[rgb]{0.73,0.13,0.13}{\textit{{#1}}}}
    \newcommand{\AnnotationTok}[1]{\textcolor[rgb]{0.38,0.63,0.69}{\textbf{\textit{{#1}}}}}
    \newcommand{\CommentVarTok}[1]{\textcolor[rgb]{0.38,0.63,0.69}{\textbf{\textit{{#1}}}}}
    \newcommand{\VariableTok}[1]{\textcolor[rgb]{0.10,0.09,0.49}{{#1}}}
    \newcommand{\ControlFlowTok}[1]{\textcolor[rgb]{0.00,0.44,0.13}{\textbf{{#1}}}}
    \newcommand{\OperatorTok}[1]{\textcolor[rgb]{0.40,0.40,0.40}{{#1}}}
    \newcommand{\BuiltInTok}[1]{{#1}}
    \newcommand{\ExtensionTok}[1]{{#1}}
    \newcommand{\PreprocessorTok}[1]{\textcolor[rgb]{0.74,0.48,0.00}{{#1}}}
    \newcommand{\AttributeTok}[1]{\textcolor[rgb]{0.49,0.56,0.16}{{#1}}}
    \newcommand{\InformationTok}[1]{\textcolor[rgb]{0.38,0.63,0.69}{\textbf{\textit{{#1}}}}}
    \newcommand{\WarningTok}[1]{\textcolor[rgb]{0.38,0.63,0.69}{\textbf{\textit{{#1}}}}}
    
    
    % Define a nice break command that doesn't care if a line doesn't already
    % exist.
    \def\br{\hspace*{\fill} \\* }
    % Math Jax compatability definitions
    \def\gt{>}
    \def\lt{<}
    % Document parameters
    \title{cagri\_aslanbas\_project}
    
    
    

    % Pygments definitions
    
\makeatletter
\def\PY@reset{\let\PY@it=\relax \let\PY@bf=\relax%
    \let\PY@ul=\relax \let\PY@tc=\relax%
    \let\PY@bc=\relax \let\PY@ff=\relax}
\def\PY@tok#1{\csname PY@tok@#1\endcsname}
\def\PY@toks#1+{\ifx\relax#1\empty\else%
    \PY@tok{#1}\expandafter\PY@toks\fi}
\def\PY@do#1{\PY@bc{\PY@tc{\PY@ul{%
    \PY@it{\PY@bf{\PY@ff{#1}}}}}}}
\def\PY#1#2{\PY@reset\PY@toks#1+\relax+\PY@do{#2}}

\expandafter\def\csname PY@tok@w\endcsname{\def\PY@tc##1{\textcolor[rgb]{0.73,0.73,0.73}{##1}}}
\expandafter\def\csname PY@tok@c\endcsname{\let\PY@it=\textit\def\PY@tc##1{\textcolor[rgb]{0.25,0.50,0.50}{##1}}}
\expandafter\def\csname PY@tok@cp\endcsname{\def\PY@tc##1{\textcolor[rgb]{0.74,0.48,0.00}{##1}}}
\expandafter\def\csname PY@tok@k\endcsname{\let\PY@bf=\textbf\def\PY@tc##1{\textcolor[rgb]{0.00,0.50,0.00}{##1}}}
\expandafter\def\csname PY@tok@kp\endcsname{\def\PY@tc##1{\textcolor[rgb]{0.00,0.50,0.00}{##1}}}
\expandafter\def\csname PY@tok@kt\endcsname{\def\PY@tc##1{\textcolor[rgb]{0.69,0.00,0.25}{##1}}}
\expandafter\def\csname PY@tok@o\endcsname{\def\PY@tc##1{\textcolor[rgb]{0.40,0.40,0.40}{##1}}}
\expandafter\def\csname PY@tok@ow\endcsname{\let\PY@bf=\textbf\def\PY@tc##1{\textcolor[rgb]{0.67,0.13,1.00}{##1}}}
\expandafter\def\csname PY@tok@nb\endcsname{\def\PY@tc##1{\textcolor[rgb]{0.00,0.50,0.00}{##1}}}
\expandafter\def\csname PY@tok@nf\endcsname{\def\PY@tc##1{\textcolor[rgb]{0.00,0.00,1.00}{##1}}}
\expandafter\def\csname PY@tok@nc\endcsname{\let\PY@bf=\textbf\def\PY@tc##1{\textcolor[rgb]{0.00,0.00,1.00}{##1}}}
\expandafter\def\csname PY@tok@nn\endcsname{\let\PY@bf=\textbf\def\PY@tc##1{\textcolor[rgb]{0.00,0.00,1.00}{##1}}}
\expandafter\def\csname PY@tok@ne\endcsname{\let\PY@bf=\textbf\def\PY@tc##1{\textcolor[rgb]{0.82,0.25,0.23}{##1}}}
\expandafter\def\csname PY@tok@nv\endcsname{\def\PY@tc##1{\textcolor[rgb]{0.10,0.09,0.49}{##1}}}
\expandafter\def\csname PY@tok@no\endcsname{\def\PY@tc##1{\textcolor[rgb]{0.53,0.00,0.00}{##1}}}
\expandafter\def\csname PY@tok@nl\endcsname{\def\PY@tc##1{\textcolor[rgb]{0.63,0.63,0.00}{##1}}}
\expandafter\def\csname PY@tok@ni\endcsname{\let\PY@bf=\textbf\def\PY@tc##1{\textcolor[rgb]{0.60,0.60,0.60}{##1}}}
\expandafter\def\csname PY@tok@na\endcsname{\def\PY@tc##1{\textcolor[rgb]{0.49,0.56,0.16}{##1}}}
\expandafter\def\csname PY@tok@nt\endcsname{\let\PY@bf=\textbf\def\PY@tc##1{\textcolor[rgb]{0.00,0.50,0.00}{##1}}}
\expandafter\def\csname PY@tok@nd\endcsname{\def\PY@tc##1{\textcolor[rgb]{0.67,0.13,1.00}{##1}}}
\expandafter\def\csname PY@tok@s\endcsname{\def\PY@tc##1{\textcolor[rgb]{0.73,0.13,0.13}{##1}}}
\expandafter\def\csname PY@tok@sd\endcsname{\let\PY@it=\textit\def\PY@tc##1{\textcolor[rgb]{0.73,0.13,0.13}{##1}}}
\expandafter\def\csname PY@tok@si\endcsname{\let\PY@bf=\textbf\def\PY@tc##1{\textcolor[rgb]{0.73,0.40,0.53}{##1}}}
\expandafter\def\csname PY@tok@se\endcsname{\let\PY@bf=\textbf\def\PY@tc##1{\textcolor[rgb]{0.73,0.40,0.13}{##1}}}
\expandafter\def\csname PY@tok@sr\endcsname{\def\PY@tc##1{\textcolor[rgb]{0.73,0.40,0.53}{##1}}}
\expandafter\def\csname PY@tok@ss\endcsname{\def\PY@tc##1{\textcolor[rgb]{0.10,0.09,0.49}{##1}}}
\expandafter\def\csname PY@tok@sx\endcsname{\def\PY@tc##1{\textcolor[rgb]{0.00,0.50,0.00}{##1}}}
\expandafter\def\csname PY@tok@m\endcsname{\def\PY@tc##1{\textcolor[rgb]{0.40,0.40,0.40}{##1}}}
\expandafter\def\csname PY@tok@gh\endcsname{\let\PY@bf=\textbf\def\PY@tc##1{\textcolor[rgb]{0.00,0.00,0.50}{##1}}}
\expandafter\def\csname PY@tok@gu\endcsname{\let\PY@bf=\textbf\def\PY@tc##1{\textcolor[rgb]{0.50,0.00,0.50}{##1}}}
\expandafter\def\csname PY@tok@gd\endcsname{\def\PY@tc##1{\textcolor[rgb]{0.63,0.00,0.00}{##1}}}
\expandafter\def\csname PY@tok@gi\endcsname{\def\PY@tc##1{\textcolor[rgb]{0.00,0.63,0.00}{##1}}}
\expandafter\def\csname PY@tok@gr\endcsname{\def\PY@tc##1{\textcolor[rgb]{1.00,0.00,0.00}{##1}}}
\expandafter\def\csname PY@tok@ge\endcsname{\let\PY@it=\textit}
\expandafter\def\csname PY@tok@gs\endcsname{\let\PY@bf=\textbf}
\expandafter\def\csname PY@tok@gp\endcsname{\let\PY@bf=\textbf\def\PY@tc##1{\textcolor[rgb]{0.00,0.00,0.50}{##1}}}
\expandafter\def\csname PY@tok@go\endcsname{\def\PY@tc##1{\textcolor[rgb]{0.53,0.53,0.53}{##1}}}
\expandafter\def\csname PY@tok@gt\endcsname{\def\PY@tc##1{\textcolor[rgb]{0.00,0.27,0.87}{##1}}}
\expandafter\def\csname PY@tok@err\endcsname{\def\PY@bc##1{\setlength{\fboxsep}{0pt}\fcolorbox[rgb]{1.00,0.00,0.00}{1,1,1}{\strut ##1}}}
\expandafter\def\csname PY@tok@kc\endcsname{\let\PY@bf=\textbf\def\PY@tc##1{\textcolor[rgb]{0.00,0.50,0.00}{##1}}}
\expandafter\def\csname PY@tok@kd\endcsname{\let\PY@bf=\textbf\def\PY@tc##1{\textcolor[rgb]{0.00,0.50,0.00}{##1}}}
\expandafter\def\csname PY@tok@kn\endcsname{\let\PY@bf=\textbf\def\PY@tc##1{\textcolor[rgb]{0.00,0.50,0.00}{##1}}}
\expandafter\def\csname PY@tok@kr\endcsname{\let\PY@bf=\textbf\def\PY@tc##1{\textcolor[rgb]{0.00,0.50,0.00}{##1}}}
\expandafter\def\csname PY@tok@bp\endcsname{\def\PY@tc##1{\textcolor[rgb]{0.00,0.50,0.00}{##1}}}
\expandafter\def\csname PY@tok@fm\endcsname{\def\PY@tc##1{\textcolor[rgb]{0.00,0.00,1.00}{##1}}}
\expandafter\def\csname PY@tok@vc\endcsname{\def\PY@tc##1{\textcolor[rgb]{0.10,0.09,0.49}{##1}}}
\expandafter\def\csname PY@tok@vg\endcsname{\def\PY@tc##1{\textcolor[rgb]{0.10,0.09,0.49}{##1}}}
\expandafter\def\csname PY@tok@vi\endcsname{\def\PY@tc##1{\textcolor[rgb]{0.10,0.09,0.49}{##1}}}
\expandafter\def\csname PY@tok@vm\endcsname{\def\PY@tc##1{\textcolor[rgb]{0.10,0.09,0.49}{##1}}}
\expandafter\def\csname PY@tok@sa\endcsname{\def\PY@tc##1{\textcolor[rgb]{0.73,0.13,0.13}{##1}}}
\expandafter\def\csname PY@tok@sb\endcsname{\def\PY@tc##1{\textcolor[rgb]{0.73,0.13,0.13}{##1}}}
\expandafter\def\csname PY@tok@sc\endcsname{\def\PY@tc##1{\textcolor[rgb]{0.73,0.13,0.13}{##1}}}
\expandafter\def\csname PY@tok@dl\endcsname{\def\PY@tc##1{\textcolor[rgb]{0.73,0.13,0.13}{##1}}}
\expandafter\def\csname PY@tok@s2\endcsname{\def\PY@tc##1{\textcolor[rgb]{0.73,0.13,0.13}{##1}}}
\expandafter\def\csname PY@tok@sh\endcsname{\def\PY@tc##1{\textcolor[rgb]{0.73,0.13,0.13}{##1}}}
\expandafter\def\csname PY@tok@s1\endcsname{\def\PY@tc##1{\textcolor[rgb]{0.73,0.13,0.13}{##1}}}
\expandafter\def\csname PY@tok@mb\endcsname{\def\PY@tc##1{\textcolor[rgb]{0.40,0.40,0.40}{##1}}}
\expandafter\def\csname PY@tok@mf\endcsname{\def\PY@tc##1{\textcolor[rgb]{0.40,0.40,0.40}{##1}}}
\expandafter\def\csname PY@tok@mh\endcsname{\def\PY@tc##1{\textcolor[rgb]{0.40,0.40,0.40}{##1}}}
\expandafter\def\csname PY@tok@mi\endcsname{\def\PY@tc##1{\textcolor[rgb]{0.40,0.40,0.40}{##1}}}
\expandafter\def\csname PY@tok@il\endcsname{\def\PY@tc##1{\textcolor[rgb]{0.40,0.40,0.40}{##1}}}
\expandafter\def\csname PY@tok@mo\endcsname{\def\PY@tc##1{\textcolor[rgb]{0.40,0.40,0.40}{##1}}}
\expandafter\def\csname PY@tok@ch\endcsname{\let\PY@it=\textit\def\PY@tc##1{\textcolor[rgb]{0.25,0.50,0.50}{##1}}}
\expandafter\def\csname PY@tok@cm\endcsname{\let\PY@it=\textit\def\PY@tc##1{\textcolor[rgb]{0.25,0.50,0.50}{##1}}}
\expandafter\def\csname PY@tok@cpf\endcsname{\let\PY@it=\textit\def\PY@tc##1{\textcolor[rgb]{0.25,0.50,0.50}{##1}}}
\expandafter\def\csname PY@tok@c1\endcsname{\let\PY@it=\textit\def\PY@tc##1{\textcolor[rgb]{0.25,0.50,0.50}{##1}}}
\expandafter\def\csname PY@tok@cs\endcsname{\let\PY@it=\textit\def\PY@tc##1{\textcolor[rgb]{0.25,0.50,0.50}{##1}}}

\def\PYZbs{\char`\\}
\def\PYZus{\char`\_}
\def\PYZob{\char`\{}
\def\PYZcb{\char`\}}
\def\PYZca{\char`\^}
\def\PYZam{\char`\&}
\def\PYZlt{\char`\<}
\def\PYZgt{\char`\>}
\def\PYZsh{\char`\#}
\def\PYZpc{\char`\%}
\def\PYZdl{\char`\$}
\def\PYZhy{\char`\-}
\def\PYZsq{\char`\'}
\def\PYZdq{\char`\"}
\def\PYZti{\char`\~}
% for compatibility with earlier versions
\def\PYZat{@}
\def\PYZlb{[}
\def\PYZrb{]}
\makeatother


    % Exact colors from NB
    \definecolor{incolor}{rgb}{0.0, 0.0, 0.5}
    \definecolor{outcolor}{rgb}{0.545, 0.0, 0.0}



    
    % Prevent overflowing lines due to hard-to-break entities
    \sloppy 
    % Setup hyperref package
    \hypersetup{
      breaklinks=true,  % so long urls are correctly broken across lines
      colorlinks=true,
      urlcolor=urlcolor,
      linkcolor=linkcolor,
      citecolor=citecolor,
      }
    % Slightly bigger margins than the latex defaults
    
    \geometry{verbose,tmargin=1in,bmargin=1in,lmargin=1in,rmargin=1in}
    
    

    \begin{document}
    
    
    \maketitle
    
    

    
    \paragraph{Student Name: Cagri
Aslanbas}\label{student-name-cagri-aslanbas}

\section{Project: Investigation of TMDb Movie Data
Set}\label{project-investigation-of-tmdb-movie-data-set}

\subsection{Table of Contents}\label{table-of-contents}

Introduction

Data Wrangling

Cleaning

Exploratory Data Analysis

Conclusions

     \#\# Introduction

\begin{quote}
This Jupyter Notebook document explores The Movie Database (TMDb) data
set. It scrutinizes the correlation between movies' vote averages and
other factors between each other such as its genre, revenue etc.

Vote average variable of the movies will be the dependent variable and
genres, revenue and release years will be the indepentent variables of
this study.

NOTE: We will use revenue\_adj column for our studies as this variable
is in terms of 2010 dollars, accounting for inflation over time, which
is more comparable.
\end{quote}

    \begin{Verbatim}[commandchars=\\\{\}]
{\color{incolor}In [{\color{incolor}1}]:} \PY{c+c1}{\PYZsh{} import the libraries that will be used}
        \PY{k+kn}{import} \PY{n+nn}{numpy} \PY{k}{as} \PY{n+nn}{np}
        \PY{k+kn}{import} \PY{n+nn}{pandas} \PY{k}{as} \PY{n+nn}{pd}
        \PY{k+kn}{import} \PY{n+nn}{matplotlib}\PY{n+nn}{.}\PY{n+nn}{pyplot} \PY{k}{as} \PY{n+nn}{plt}
        \PY{k+kn}{import} \PY{n+nn}{seaborn} \PY{k}{as} \PY{n+nn}{sns}
        
        \PY{o}{\PYZpc{}}\PY{k}{matplotlib} inline
\end{Verbatim}


     \#\# Data Wrangling

\begin{quote}
In this section, we will have a look at the raw TMDb data and shape it
into a form such that it will have the relevant data that will be needed
in our analysis only.

Colunms that won't be used will be trimmed in this section.
\end{quote}

\subsubsection{General Properties:}\label{general-properties}

    \begin{Verbatim}[commandchars=\\\{\}]
{\color{incolor}In [{\color{incolor}2}]:} \PY{c+c1}{\PYZsh{} read the TMDb csv file}
        \PY{n}{df} \PY{o}{=} \PY{n}{pd}\PY{o}{.}\PY{n}{read\PYZus{}csv}\PY{p}{(}\PY{l+s+s2}{\PYZdq{}}\PY{l+s+s2}{tmdb\PYZhy{}movies.csv}\PY{l+s+s2}{\PYZdq{}}\PY{p}{)}
        \PY{n}{df}\PY{o}{.}\PY{n}{info}\PY{p}{(}\PY{p}{)}
\end{Verbatim}


    \begin{Verbatim}[commandchars=\\\{\}]
<class 'pandas.core.frame.DataFrame'>
RangeIndex: 10866 entries, 0 to 10865
Data columns (total 21 columns):
id                      10866 non-null int64
imdb\_id                 10856 non-null object
popularity              10866 non-null float64
budget                  10866 non-null int64
revenue                 10866 non-null int64
original\_title          10866 non-null object
cast                    10790 non-null object
homepage                2936 non-null object
director                10822 non-null object
tagline                 8042 non-null object
keywords                9373 non-null object
overview                10862 non-null object
runtime                 10866 non-null int64
genres                  10843 non-null object
production\_companies    9836 non-null object
release\_date            10866 non-null object
vote\_count              10866 non-null int64
vote\_average            10866 non-null float64
release\_year            10866 non-null int64
budget\_adj              10866 non-null float64
revenue\_adj             10866 non-null float64
dtypes: float64(4), int64(6), object(11)
memory usage: 1.7+ MB

    \end{Verbatim}

    \subsubsection{Trimming columns that will not be
used:}\label{trimming-columns-that-will-not-be-used}

\begin{quote}
This study will get use of columns vote\_average (dependant variable),
revenue\_adj, release\_year and genres (independent variables) for
exploration. We will also need the original\_title, id and vote\_count
columns for clear identification of the movies.

Therefore, other columns will be dropped before data cleaning.
\end{quote}

    \begin{Verbatim}[commandchars=\\\{\}]
{\color{incolor}In [{\color{incolor}3}]:} \PY{c+c1}{\PYZsh{} dropping redundant columns}
        \PY{n}{df}\PY{o}{.}\PY{n}{drop}\PY{p}{(}\PY{p}{[}\PY{l+s+s1}{\PYZsq{}}\PY{l+s+s1}{popularity}\PY{l+s+s1}{\PYZsq{}}\PY{p}{,} \PY{l+s+s1}{\PYZsq{}}\PY{l+s+s1}{budget}\PY{l+s+s1}{\PYZsq{}}\PY{p}{,} \PY{l+s+s1}{\PYZsq{}}\PY{l+s+s1}{revenue}\PY{l+s+s1}{\PYZsq{}}\PY{p}{,} \PY{l+s+s1}{\PYZsq{}}\PY{l+s+s1}{runtime}\PY{l+s+s1}{\PYZsq{}}\PY{p}{,} \PY{l+s+s1}{\PYZsq{}}\PY{l+s+s1}{budget\PYZus{}adj}\PY{l+s+s1}{\PYZsq{}}\PY{p}{,} \PY{l+s+s1}{\PYZsq{}}\PY{l+s+s1}{director}\PY{l+s+s1}{\PYZsq{}}\PY{p}{,} \PY{l+s+s1}{\PYZsq{}}\PY{l+s+s1}{cast}\PY{l+s+s1}{\PYZsq{}}\PY{p}{,} \PY{l+s+s1}{\PYZsq{}}\PY{l+s+s1}{keywords}\PY{l+s+s1}{\PYZsq{}}\PY{p}{,} \PY{l+s+s1}{\PYZsq{}}\PY{l+s+s1}{release\PYZus{}date}\PY{l+s+s1}{\PYZsq{}}\PY{p}{,} \PY{l+s+s1}{\PYZsq{}}\PY{l+s+s1}{homepage}\PY{l+s+s1}{\PYZsq{}}\PY{p}{,} \PY{l+s+s1}{\PYZsq{}}\PY{l+s+s1}{imdb\PYZus{}id}\PY{l+s+s1}{\PYZsq{}}\PY{p}{,} \PY{l+s+s1}{\PYZsq{}}\PY{l+s+s1}{tagline}\PY{l+s+s1}{\PYZsq{}}\PY{p}{,} \PY{l+s+s1}{\PYZsq{}}\PY{l+s+s1}{overview}\PY{l+s+s1}{\PYZsq{}}\PY{p}{,} \PY{l+s+s1}{\PYZsq{}}\PY{l+s+s1}{production\PYZus{}companies}\PY{l+s+s1}{\PYZsq{}}\PY{p}{]}\PY{p}{,} \PY{n}{axis}\PY{o}{=}\PY{l+m+mi}{1}\PY{p}{,} \PY{n}{inplace}\PY{o}{=}\PY{k+kc}{True}\PY{p}{)}
        \PY{n}{df}\PY{o}{.}\PY{n}{info}\PY{p}{(}\PY{p}{)}
\end{Verbatim}


    \begin{Verbatim}[commandchars=\\\{\}]
<class 'pandas.core.frame.DataFrame'>
RangeIndex: 10866 entries, 0 to 10865
Data columns (total 7 columns):
id                10866 non-null int64
original\_title    10866 non-null object
genres            10843 non-null object
vote\_count        10866 non-null int64
vote\_average      10866 non-null float64
release\_year      10866 non-null int64
revenue\_adj       10866 non-null float64
dtypes: float64(2), int64(3), object(2)
memory usage: 594.3+ KB

    \end{Verbatim}

    \begin{quote}
We can see that there are 10866 rows. Except genres column, none of the
columns have null values. We need to drop the rows that has null genres
value as we cannot replace null genres values with something else (it is
not a measurable variable).
\end{quote}

    \begin{Verbatim}[commandchars=\\\{\}]
{\color{incolor}In [{\color{incolor}4}]:} \PY{c+c1}{\PYZsh{} drop rows with null values}
        \PY{n}{df}\PY{o}{.}\PY{n}{dropna}\PY{p}{(}\PY{n}{how}\PY{o}{=}\PY{l+s+s1}{\PYZsq{}}\PY{l+s+s1}{any}\PY{l+s+s1}{\PYZsq{}}\PY{p}{,} \PY{n}{axis}\PY{o}{=}\PY{l+m+mi}{0}\PY{p}{,} \PY{n}{inplace}\PY{o}{=}\PY{k+kc}{True}\PY{p}{)}
        \PY{n}{df}\PY{o}{.}\PY{n}{info}\PY{p}{(}\PY{p}{)}
\end{Verbatim}


    \begin{Verbatim}[commandchars=\\\{\}]
<class 'pandas.core.frame.DataFrame'>
Int64Index: 10843 entries, 0 to 10865
Data columns (total 7 columns):
id                10843 non-null int64
original\_title    10843 non-null object
genres            10843 non-null object
vote\_count        10843 non-null int64
vote\_average      10843 non-null float64
release\_year      10843 non-null int64
revenue\_adj       10843 non-null float64
dtypes: float64(2), int64(3), object(2)
memory usage: 677.7+ KB

    \end{Verbatim}

     \#\# Cleaning

\begin{quote}
In this section, we will clean the cells that are null or zero and also
irreplaceable.

In the end, we will make sure we have meaningful data in each movie
entry.
\end{quote}

    \begin{Verbatim}[commandchars=\\\{\}]
{\color{incolor}In [{\color{incolor}5}]:} \PY{n}{df}\PY{o}{.}\PY{n}{describe}\PY{p}{(}\PY{p}{)}
\end{Verbatim}


\begin{Verbatim}[commandchars=\\\{\}]
{\color{outcolor}Out[{\color{outcolor}5}]:}                   id    vote\_count  vote\_average  release\_year   revenue\_adj
        count   10843.000000  10843.000000  10843.000000  10843.000000  1.084300e+04
        mean    65868.491930    217.813705      5.973974   2001.315595  5.147332e+07
        std     91977.394803    576.155351      0.934260     12.813298  1.447664e+08
        min         5.000000     10.000000      1.500000   1960.000000  0.000000e+00
        25\%     10589.500000     17.000000      5.400000   1995.000000  0.000000e+00
        50\%     20558.000000     38.000000      6.000000   2006.000000  0.000000e+00
        75\%     75182.000000    146.000000      6.600000   2011.000000  3.387655e+07
        max    417859.000000   9767.000000      9.200000   2015.000000  2.827124e+09
\end{Verbatim}
            
    \begin{quote}
It's pretty clear that at least half (50\% percentile) of the movie
rows' revenue\_adj values are equal to zero (0.0). In order to make
healthy analysis, we need rows with non-zero revenue\_adj values.
Therefore I will filter those with query function of pandas.
\end{quote}

    \begin{Verbatim}[commandchars=\\\{\}]
{\color{incolor}In [{\color{incolor}6}]:} \PY{c+c1}{\PYZsh{} drop rows that have zero revenue\PYZus{}adj values}
        \PY{n}{df} \PY{o}{=} \PY{n}{df}\PY{o}{.}\PY{n}{query}\PY{p}{(}\PY{l+s+s1}{\PYZsq{}}\PY{l+s+s1}{revenue\PYZus{}adj \PYZgt{} 0.0}\PY{l+s+s1}{\PYZsq{}}\PY{p}{)}
        \PY{n}{df}\PY{o}{.}\PY{n}{info}\PY{p}{(}\PY{p}{)}
\end{Verbatim}


    \begin{Verbatim}[commandchars=\\\{\}]
<class 'pandas.core.frame.DataFrame'>
Int64Index: 4850 entries, 0 to 10848
Data columns (total 7 columns):
id                4850 non-null int64
original\_title    4850 non-null object
genres            4850 non-null object
vote\_count        4850 non-null int64
vote\_average      4850 non-null float64
release\_year      4850 non-null int64
revenue\_adj       4850 non-null float64
dtypes: float64(2), int64(3), object(2)
memory usage: 303.1+ KB

    \end{Verbatim}

    \begin{Verbatim}[commandchars=\\\{\}]
{\color{incolor}In [{\color{incolor}7}]:} \PY{n}{df}\PY{o}{.}\PY{n}{describe}\PY{p}{(}\PY{p}{)}
\end{Verbatim}


\begin{Verbatim}[commandchars=\\\{\}]
{\color{outcolor}Out[{\color{outcolor}7}]:}                   id   vote\_count  vote\_average  release\_year   revenue\_adj
        count    4850.000000  4850.000000   4850.000000   4850.000000  4.850000e+03
        mean    44575.000619   436.215876      6.148763   2000.921649  1.150774e+08
        std     72361.405911   806.416200      0.798795     11.569192  1.988419e+08
        min         5.000000    10.000000      2.100000   1960.000000  2.370705e+00
        25\%      8286.000000    46.000000      5.600000   1994.000000  1.046262e+07
        50\%     12154.500000   147.000000      6.200000   2004.000000  4.392749e+07
        75\%     43956.500000   435.000000      6.700000   2010.000000  1.315644e+08
        max    417859.000000  9767.000000      8.400000   2015.000000  2.827124e+09
\end{Verbatim}
            
    \begin{quote}
In order to make a fair analysis, movies that have low number of
vote\_counts need to be eliminated. In the next step, I will dig deep
into the vote\_count column and drop some movies that have vote counts
under a certain threshold. To do that lets start with the histogram of
the vote\_counts.
\end{quote}

    \begin{Verbatim}[commandchars=\\\{\}]
{\color{incolor}In [{\color{incolor}8}]:} \PY{n}{df}\PY{p}{[}\PY{l+s+s1}{\PYZsq{}}\PY{l+s+s1}{vote\PYZus{}count}\PY{l+s+s1}{\PYZsq{}}\PY{p}{]}\PY{o}{.}\PY{n}{hist}\PY{p}{(}\PY{n}{figsize}\PY{o}{=}\PY{p}{(}\PY{l+m+mi}{20}\PY{p}{,}\PY{l+m+mi}{7}\PY{p}{)}\PY{p}{)}\PY{p}{;}
\end{Verbatim}


    \begin{center}
    \adjustimage{max size={0.9\linewidth}{0.9\paperheight}}{output_15_0.png}
    \end{center}
    { \hspace*{\fill} \\}
    
    \begin{quote}
Let's split the movies into two groups: movies that has total vote count
under the mean value (which is 217) and over the mean value. And see
both of these data frames correlation to the vote\_average.
\end{quote}

    \begin{Verbatim}[commandchars=\\\{\}]
{\color{incolor}In [{\color{incolor}9}]:} \PY{n}{df\PYZus{}low\PYZus{}vote\PYZus{}counts} \PY{o}{=} \PY{n}{df}\PY{o}{.}\PY{n}{query}\PY{p}{(}\PY{l+s+s1}{\PYZsq{}}\PY{l+s+s1}{vote\PYZus{}count \PYZlt{}= 217}\PY{l+s+s1}{\PYZsq{}}\PY{p}{)}\PY{p}{;}
        \PY{n}{df\PYZus{}high\PYZus{}vote\PYZus{}counts} \PY{o}{=} \PY{n}{df}\PY{o}{.}\PY{n}{query}\PY{p}{(}\PY{l+s+s1}{\PYZsq{}}\PY{l+s+s1}{vote\PYZus{}count \PYZgt{} 217}\PY{l+s+s1}{\PYZsq{}}\PY{p}{)}\PY{p}{;}
        \PY{n}{df\PYZus{}low\PYZus{}vote\PYZus{}counts}\PY{o}{.}\PY{n}{plot}\PY{p}{(}\PY{n}{x} \PY{o}{=} \PY{l+s+s1}{\PYZsq{}}\PY{l+s+s1}{vote\PYZus{}count}\PY{l+s+s1}{\PYZsq{}}\PY{p}{,} \PY{n}{y} \PY{o}{=} \PY{l+s+s1}{\PYZsq{}}\PY{l+s+s1}{vote\PYZus{}average}\PY{l+s+s1}{\PYZsq{}}\PY{p}{,} \PY{n}{kind}\PY{o}{=}\PY{l+s+s1}{\PYZsq{}}\PY{l+s+s1}{scatter}\PY{l+s+s1}{\PYZsq{}}\PY{p}{,} \PY{n}{figsize}\PY{o}{=}\PY{p}{(}\PY{l+m+mi}{8}\PY{p}{,}\PY{l+m+mi}{8}\PY{p}{)}\PY{p}{)}\PY{p}{;}
        \PY{n}{df\PYZus{}high\PYZus{}vote\PYZus{}counts}\PY{o}{.}\PY{n}{plot}\PY{p}{(}\PY{n}{x} \PY{o}{=} \PY{l+s+s1}{\PYZsq{}}\PY{l+s+s1}{vote\PYZus{}count}\PY{l+s+s1}{\PYZsq{}}\PY{p}{,} \PY{n}{y} \PY{o}{=} \PY{l+s+s1}{\PYZsq{}}\PY{l+s+s1}{vote\PYZus{}average}\PY{l+s+s1}{\PYZsq{}}\PY{p}{,} \PY{n}{kind}\PY{o}{=}\PY{l+s+s1}{\PYZsq{}}\PY{l+s+s1}{scatter}\PY{l+s+s1}{\PYZsq{}}\PY{p}{,} \PY{n}{figsize}\PY{o}{=}\PY{p}{(}\PY{l+m+mi}{8}\PY{p}{,}\PY{l+m+mi}{8}\PY{p}{)}\PY{p}{)}\PY{p}{;}
\end{Verbatim}


    \begin{center}
    \adjustimage{max size={0.9\linewidth}{0.9\paperheight}}{output_17_0.png}
    \end{center}
    { \hspace*{\fill} \\}
    
    \begin{center}
    \adjustimage{max size={0.9\linewidth}{0.9\paperheight}}{output_17_1.png}
    \end{center}
    { \hspace*{\fill} \\}
    
    \begin{quote}
When these two graphs are compared, it's obvious that the vote average
of movies having vote counts less than 45-50 do not have a meaningful
pattern as low vote counts can shape the vote average dramatically.
Therefore we can drop the movies with vote counts less than 45.
\end{quote}

    \begin{Verbatim}[commandchars=\\\{\}]
{\color{incolor}In [{\color{incolor}10}]:} \PY{n}{df} \PY{o}{=} \PY{n}{df}\PY{o}{.}\PY{n}{query}\PY{p}{(}\PY{l+s+s1}{\PYZsq{}}\PY{l+s+s1}{vote\PYZus{}count \PYZgt{}= 45}\PY{l+s+s1}{\PYZsq{}}\PY{p}{)}
         \PY{n}{df}\PY{o}{.}\PY{n}{info}\PY{p}{(}\PY{p}{)}
\end{Verbatim}


    \begin{Verbatim}[commandchars=\\\{\}]
<class 'pandas.core.frame.DataFrame'>
Int64Index: 3689 entries, 0 to 10828
Data columns (total 7 columns):
id                3689 non-null int64
original\_title    3689 non-null object
genres            3689 non-null object
vote\_count        3689 non-null int64
vote\_average      3689 non-null float64
release\_year      3689 non-null int64
revenue\_adj       3689 non-null float64
dtypes: float64(2), int64(3), object(2)
memory usage: 230.6+ KB

    \end{Verbatim}

    \begin{Verbatim}[commandchars=\\\{\}]
{\color{incolor}In [{\color{incolor}11}]:} \PY{c+c1}{\PYZsh{} lets see the first 10 rows}
         \PY{n}{df}\PY{o}{.}\PY{n}{head}\PY{p}{(}\PY{l+m+mi}{10}\PY{p}{)}
\end{Verbatim}


\begin{Verbatim}[commandchars=\\\{\}]
{\color{outcolor}Out[{\color{outcolor}11}]:}        id                original\_title  \textbackslash{}
         0  135397                Jurassic World   
         1   76341            Mad Max: Fury Road   
         2  262500                     Insurgent   
         3  140607  Star Wars: The Force Awakens   
         4  168259                     Furious 7   
         5  281957                  The Revenant   
         6   87101            Terminator Genisys   
         7  286217                   The Martian   
         8  211672                       Minions   
         9  150540                    Inside Out   
         
                                               genres  vote\_count  vote\_average  \textbackslash{}
         0  Action|Adventure|Science Fiction|Thriller        5562           6.5   
         1  Action|Adventure|Science Fiction|Thriller        6185           7.1   
         2         Adventure|Science Fiction|Thriller        2480           6.3   
         3   Action|Adventure|Science Fiction|Fantasy        5292           7.5   
         4                      Action|Crime|Thriller        2947           7.3   
         5           Western|Drama|Adventure|Thriller        3929           7.2   
         6  Science Fiction|Action|Thriller|Adventure        2598           5.8   
         7            Drama|Adventure|Science Fiction        4572           7.6   
         8          Family|Animation|Adventure|Comedy        2893           6.5   
         9                    Comedy|Animation|Family        3935           8.0   
         
            release\_year   revenue\_adj  
         0          2015  1.392446e+09  
         1          2015  3.481613e+08  
         2          2015  2.716190e+08  
         3          2015  1.902723e+09  
         4          2015  1.385749e+09  
         5          2015  4.903142e+08  
         6          2015  4.053551e+08  
         7          2015  5.477497e+08  
         8          2015  1.064192e+09  
         9          2015  7.854116e+08  
\end{Verbatim}
            
    \begin{quote}
Finally, we need to drop the duplicate values if any.
\end{quote}

    \begin{Verbatim}[commandchars=\\\{\}]
{\color{incolor}In [{\color{incolor}12}]:} \PY{n+nb}{sum}\PY{p}{(}\PY{n}{df}\PY{o}{.}\PY{n}{duplicated}\PY{p}{(}\PY{p}{)}\PY{p}{)}
\end{Verbatim}


\begin{Verbatim}[commandchars=\\\{\}]
{\color{outcolor}Out[{\color{outcolor}12}]:} 1
\end{Verbatim}
            
    \begin{Verbatim}[commandchars=\\\{\}]
{\color{incolor}In [{\color{incolor}13}]:} \PY{n}{df}\PY{o}{.}\PY{n}{drop\PYZus{}duplicates}\PY{p}{(}\PY{n}{inplace}\PY{o}{=}\PY{k+kc}{True}\PY{p}{)}
         \PY{n+nb}{sum}\PY{p}{(}\PY{n}{df}\PY{o}{.}\PY{n}{duplicated}\PY{p}{(}\PY{p}{)}\PY{p}{)}
\end{Verbatim}


\begin{Verbatim}[commandchars=\\\{\}]
{\color{outcolor}Out[{\color{outcolor}13}]:} 0
\end{Verbatim}
            
     \#\# Exploratory Data Analysis

\paragraph{In this section, we will try to find answers to two
questions:}\label{in-this-section-we-will-try-to-find-answers-to-two-questions}

Which genres has the highest average of votes from year to year in
2000s?

Is there any relation between the movies' vote average and revenue in
terms of 2010 dollars?

    \subparagraph{1. Which genres has the highest average of votes from year
to year in
2000s?}\label{which-genres-has-the-highest-average-of-votes-from-year-to-year-in-2000s}

\begin{quote}
This question requires a multiple-variable (2d) exploration, which
consists of a correlation between our dependant variable vote\_average
and two independant variables release\_year and genres.
\end{quote}

    \begin{Verbatim}[commandchars=\\\{\}]
{\color{incolor}In [{\color{incolor}14}]:} \PY{n}{df}\PY{o}{.}\PY{n}{head}\PY{p}{(}\PY{l+m+mi}{10}\PY{p}{)}
\end{Verbatim}


\begin{Verbatim}[commandchars=\\\{\}]
{\color{outcolor}Out[{\color{outcolor}14}]:}        id                original\_title  \textbackslash{}
         0  135397                Jurassic World   
         1   76341            Mad Max: Fury Road   
         2  262500                     Insurgent   
         3  140607  Star Wars: The Force Awakens   
         4  168259                     Furious 7   
         5  281957                  The Revenant   
         6   87101            Terminator Genisys   
         7  286217                   The Martian   
         8  211672                       Minions   
         9  150540                    Inside Out   
         
                                               genres  vote\_count  vote\_average  \textbackslash{}
         0  Action|Adventure|Science Fiction|Thriller        5562           6.5   
         1  Action|Adventure|Science Fiction|Thriller        6185           7.1   
         2         Adventure|Science Fiction|Thriller        2480           6.3   
         3   Action|Adventure|Science Fiction|Fantasy        5292           7.5   
         4                      Action|Crime|Thriller        2947           7.3   
         5           Western|Drama|Adventure|Thriller        3929           7.2   
         6  Science Fiction|Action|Thriller|Adventure        2598           5.8   
         7            Drama|Adventure|Science Fiction        4572           7.6   
         8          Family|Animation|Adventure|Comedy        2893           6.5   
         9                    Comedy|Animation|Family        3935           8.0   
         
            release\_year   revenue\_adj  
         0          2015  1.392446e+09  
         1          2015  3.481613e+08  
         2          2015  2.716190e+08  
         3          2015  1.902723e+09  
         4          2015  1.385749e+09  
         5          2015  4.903142e+08  
         6          2015  4.053551e+08  
         7          2015  5.477497e+08  
         8          2015  1.064192e+09  
         9          2015  7.854116e+08  
\end{Verbatim}
            
    \begin{quote}
Movies mostly have more than one genres splitted with the pipe delimiter
"\textbar{}" character. We need to create duplicates of each movie by
the number of genre type it has. Then split the genres and spread them
into each duplicate movie it belongs.

For example for the movie "Jurassic World" (id:135397), it has 4 genres,
namely "Action\textbar{}Adventure\textbar{}Science
Fiction\textbar{}Thriller". We need to create 4 duplicates of "Jurassic
Park" and assign Action, Adventure, Science Fiction and Thriller to each
of its genre, separately.

Let's see how many movies have multiple genres:
\end{quote}

    \begin{Verbatim}[commandchars=\\\{\}]
{\color{incolor}In [{\color{incolor}15}]:} \PY{c+c1}{\PYZsh{} see how many movies have multiple genres}
         \PY{n}{multiple\PYZus{}genres} \PY{o}{=} \PY{n}{df}\PY{p}{[}\PY{n}{df}\PY{p}{[}\PY{l+s+s1}{\PYZsq{}}\PY{l+s+s1}{genres}\PY{l+s+s1}{\PYZsq{}}\PY{p}{]}\PY{o}{.}\PY{n}{str}\PY{o}{.}\PY{n}{contains}\PY{p}{(}\PY{l+s+s2}{\PYZdq{}}\PY{l+s+s2}{\PYZbs{}}\PY{l+s+s2}{|}\PY{l+s+s2}{\PYZdq{}}\PY{p}{)}\PY{p}{]}
         \PY{n+nb}{print}\PY{p}{(}\PY{n}{multiple\PYZus{}genres}\PY{o}{.}\PY{n}{shape}\PY{p}{)}\PY{p}{;}
         \PY{n+nb}{print}\PY{p}{(}\PY{n}{df}\PY{o}{.}\PY{n}{shape}\PY{p}{)}\PY{p}{;}
\end{Verbatim}


    \begin{Verbatim}[commandchars=\\\{\}]
(3057, 7)
(3688, 7)

    \end{Verbatim}

    \begin{quote}
Okay, 3058 of our total 3689 movies have multiple genres. Another tricky
part is that the number of genre types for each movie is not static, it
differs from one movie to another.

Since the number of genres each movie has is dynamic, I have searched a
way to do it from Stackoverflow and found a function that solves this
problem (Stackoverflow source):
\end{quote}

    \begin{Verbatim}[commandchars=\\\{\}]
{\color{incolor}In [{\color{incolor}16}]:} \PY{k}{def} \PY{n+nf}{tidy\PYZus{}split}\PY{p}{(}\PY{n}{df}\PY{p}{,} \PY{n}{column}\PY{p}{,} \PY{n}{sep}\PY{o}{=}\PY{l+s+s1}{\PYZsq{}}\PY{l+s+s1}{|}\PY{l+s+s1}{\PYZsq{}}\PY{p}{,} \PY{n}{keep}\PY{o}{=}\PY{k+kc}{False}\PY{p}{)}\PY{p}{:}
             \PY{n}{indexes} \PY{o}{=} \PY{n+nb}{list}\PY{p}{(}\PY{p}{)}
             \PY{n}{new\PYZus{}values} \PY{o}{=} \PY{n+nb}{list}\PY{p}{(}\PY{p}{)}
             \PY{n}{df} \PY{o}{=} \PY{n}{df}\PY{o}{.}\PY{n}{dropna}\PY{p}{(}\PY{n}{subset}\PY{o}{=}\PY{p}{[}\PY{n}{column}\PY{p}{]}\PY{p}{)}
             \PY{k}{for} \PY{n}{i}\PY{p}{,} \PY{n}{presplit} \PY{o+ow}{in} \PY{n+nb}{enumerate}\PY{p}{(}\PY{n}{df}\PY{p}{[}\PY{n}{column}\PY{p}{]}\PY{o}{.}\PY{n}{astype}\PY{p}{(}\PY{n+nb}{str}\PY{p}{)}\PY{p}{)}\PY{p}{:}
                 \PY{n}{values} \PY{o}{=} \PY{n}{presplit}\PY{o}{.}\PY{n}{split}\PY{p}{(}\PY{n}{sep}\PY{p}{)}
                 \PY{k}{if} \PY{n}{keep} \PY{o+ow}{and} \PY{n+nb}{len}\PY{p}{(}\PY{n}{values}\PY{p}{)} \PY{o}{\PYZgt{}} \PY{l+m+mi}{1}\PY{p}{:}
                     \PY{n}{indexes}\PY{o}{.}\PY{n}{append}\PY{p}{(}\PY{n}{i}\PY{p}{)}
                     \PY{n}{new\PYZus{}values}\PY{o}{.}\PY{n}{append}\PY{p}{(}\PY{n}{presplit}\PY{p}{)}
                 \PY{k}{for} \PY{n}{value} \PY{o+ow}{in} \PY{n}{values}\PY{p}{:}
                     \PY{n}{indexes}\PY{o}{.}\PY{n}{append}\PY{p}{(}\PY{n}{i}\PY{p}{)}
                     \PY{n}{new\PYZus{}values}\PY{o}{.}\PY{n}{append}\PY{p}{(}\PY{n}{value}\PY{p}{)}
             \PY{n}{new\PYZus{}df} \PY{o}{=} \PY{n}{df}\PY{o}{.}\PY{n}{iloc}\PY{p}{[}\PY{n}{indexes}\PY{p}{,} \PY{p}{:}\PY{p}{]}\PY{o}{.}\PY{n}{copy}\PY{p}{(}\PY{p}{)}
             \PY{n}{new\PYZus{}df}\PY{p}{[}\PY{n}{column}\PY{p}{]} \PY{o}{=} \PY{n}{new\PYZus{}values}
             \PY{k}{return} \PY{n}{new\PYZus{}df}
\end{Verbatim}


    \begin{quote}
Firstly, I created a copy of our dataframe, named df\_genres\_ops. By
using tidy\_split function I did split each genre and assign to its
corresponding duplicate movie row, named the result as
multiple\_genres\_separated.

Then I dropped the movies with multiple genres from the original copy
df\_genres\_ops.

Finally, I merged two dataframes, df\_genres\_ops and
multiple\_genres\_separated. (Udacity source)
\end{quote}

    \begin{Verbatim}[commandchars=\\\{\}]
{\color{incolor}In [{\color{incolor}18}]:} \PY{n}{df\PYZus{}genres\PYZus{}ops} \PY{o}{=} \PY{n}{df}\PY{o}{.}\PY{n}{copy}\PY{p}{(}\PY{p}{)}
         \PY{n}{multiple\PYZus{}genres\PYZus{}separated} \PY{o}{=} \PY{n}{tidy\PYZus{}split}\PY{p}{(}\PY{n}{multiple\PYZus{}genres}\PY{p}{,} \PY{l+s+s1}{\PYZsq{}}\PY{l+s+s1}{genres}\PY{l+s+s1}{\PYZsq{}}\PY{p}{,} \PY{n}{sep}\PY{o}{=}\PY{l+s+s1}{\PYZsq{}}\PY{l+s+s1}{|}\PY{l+s+s1}{\PYZsq{}}\PY{p}{)}
         \PY{n}{df\PYZus{}genres\PYZus{}ops}\PY{o}{.}\PY{n}{drop}\PY{p}{(}\PY{n}{multiple\PYZus{}genres}\PY{o}{.}\PY{n}{index}\PY{p}{,} \PY{n}{inplace}\PY{o}{=}\PY{k+kc}{True}\PY{p}{)}
         \PY{n}{df\PYZus{}genres\PYZus{}ops} \PY{o}{=} \PY{n}{df\PYZus{}genres\PYZus{}ops}\PY{o}{.}\PY{n}{append}\PY{p}{(}\PY{n}{multiple\PYZus{}genres\PYZus{}separated}\PY{p}{,} \PY{n}{ignore\PYZus{}index}\PY{o}{=}\PY{k+kc}{True}\PY{p}{)}
         \PY{n+nb}{print}\PY{p}{(}\PY{n}{df\PYZus{}genres\PYZus{}ops}\PY{o}{.}\PY{n}{shape}\PY{p}{)}
         \PY{n}{df\PYZus{}genres\PYZus{}ops}\PY{o}{.}\PY{n}{tail}\PY{p}{(}\PY{l+m+mi}{10}\PY{p}{)}
\end{Verbatim}


    \begin{Verbatim}[commandchars=\\\{\}]
(9832, 7)

    \end{Verbatim}

\begin{Verbatim}[commandchars=\\\{\}]
{\color{outcolor}Out[{\color{outcolor}18}]:}          id                  original\_title     genres  vote\_count  \textbackslash{}
         9822  11850  Invasion of the Body Snatchers    Mystery          87   
         9823  10766                 Damien: Omen II     Action          71   
         9824  10766                 Damien: Omen II      Drama          71   
         9825  10766                 Damien: Omen II     Horror          71   
         9826  10766                 Damien: Omen II   Thriller          71   
         9827  11837                  Watership Down  Adventure          67   
         9828  11837                  Watership Down  Animation          67   
         9829  11837                  Watership Down      Drama          67   
         9830   5780                    Torn Curtain    Mystery          46   
         9831   5780                    Torn Curtain   Thriller          46   
         
               vote\_average  release\_year   revenue\_adj  
         9822           6.5          1978  8.038036e+07  
         9823           6.1          1978  8.864292e+07  
         9824           6.1          1978  8.864292e+07  
         9825           6.1          1978  8.864292e+07  
         9826           6.1          1978  8.864292e+07  
         9827           6.8          1978  1.241401e+07  
         9828           6.8          1978  1.241401e+07  
         9829           6.8          1978  1.241401e+07  
         9830           6.3          1966  8.733419e+07  
         9831           6.3          1966  8.733419e+07  
\end{Verbatim}
            
    \begin{quote}
As can be seen, the movies with multiple genres are splitted and each of
its genre is assigned to one of these duplicates.
\end{quote}

    \begin{Verbatim}[commandchars=\\\{\}]
{\color{incolor}In [{\color{incolor}20}]:} \PY{c+c1}{\PYZsh{} guarantee that each genre is lowercase}
         \PY{n}{df\PYZus{}genres\PYZus{}ops}\PY{p}{[}\PY{l+s+s1}{\PYZsq{}}\PY{l+s+s1}{genres}\PY{l+s+s1}{\PYZsq{}}\PY{p}{]} \PY{o}{=} \PY{n}{df\PYZus{}genres\PYZus{}ops}\PY{p}{[}\PY{l+s+s1}{\PYZsq{}}\PY{l+s+s1}{genres}\PY{l+s+s1}{\PYZsq{}}\PY{p}{]}\PY{o}{.}\PY{n}{str}\PY{o}{.}\PY{n}{lower}\PY{p}{(}\PY{p}{)}
         \PY{n}{df\PYZus{}genres\PYZus{}ops}\PY{o}{.}\PY{n}{tail}\PY{p}{(}\PY{l+m+mi}{10}\PY{p}{)}
\end{Verbatim}


\begin{Verbatim}[commandchars=\\\{\}]
{\color{outcolor}Out[{\color{outcolor}20}]:}          id                  original\_title     genres  vote\_count  \textbackslash{}
         9822  11850  Invasion of the Body Snatchers    mystery          87   
         9823  10766                 Damien: Omen II     action          71   
         9824  10766                 Damien: Omen II      drama          71   
         9825  10766                 Damien: Omen II     horror          71   
         9826  10766                 Damien: Omen II   thriller          71   
         9827  11837                  Watership Down  adventure          67   
         9828  11837                  Watership Down  animation          67   
         9829  11837                  Watership Down      drama          67   
         9830   5780                    Torn Curtain    mystery          46   
         9831   5780                    Torn Curtain   thriller          46   
         
               vote\_average  release\_year   revenue\_adj  
         9822           6.5          1978  8.038036e+07  
         9823           6.1          1978  8.864292e+07  
         9824           6.1          1978  8.864292e+07  
         9825           6.1          1978  8.864292e+07  
         9826           6.1          1978  8.864292e+07  
         9827           6.8          1978  1.241401e+07  
         9828           6.8          1978  1.241401e+07  
         9829           6.8          1978  1.241401e+07  
         9830           6.3          1966  8.733419e+07  
         9831           6.3          1966  8.733419e+07  
\end{Verbatim}
            
    \begin{quote}
First of all, as we will be looking for the movies in 2000s, I am going
to filter df\_genres\_ops dataframe.
\end{quote}

    \begin{Verbatim}[commandchars=\\\{\}]
{\color{incolor}In [{\color{incolor}21}]:} \PY{n}{df\PYZus{}genres\PYZus{}ops} \PY{o}{=} \PY{n}{df\PYZus{}genres\PYZus{}ops}\PY{o}{.}\PY{n}{query}\PY{p}{(}\PY{l+s+s1}{\PYZsq{}}\PY{l+s+s1}{release\PYZus{}year \PYZgt{}= 2000}\PY{l+s+s1}{\PYZsq{}}\PY{p}{)}
         \PY{n}{df\PYZus{}genres\PYZus{}ops}\PY{o}{.}\PY{n}{describe}\PY{p}{(}\PY{p}{)}
\end{Verbatim}


\begin{Verbatim}[commandchars=\\\{\}]
{\color{outcolor}Out[{\color{outcolor}21}]:}                   id   vote\_count  vote\_average  release\_year   revenue\_adj
         count    6565.000000  6565.000000   6565.000000   6565.000000  6.565000e+03
         mean    60038.902513   724.939680      6.150327   2008.359634  1.444381e+08
         std     82262.182890  1053.905171      0.750845      4.376059  2.193520e+08
         min        12.000000    45.000000      3.300000   2000.000000  2.370705e+00
         25\%      8968.000000   138.000000      5.700000   2005.000000  1.812426e+07
         50\%     14873.000000   320.000000      6.100000   2009.000000  6.680736e+07
         75\%     76543.000000   823.000000      6.600000   2012.000000  1.739116e+08
         max    417859.000000  9767.000000      8.200000   2015.000000  2.827124e+09
\end{Verbatim}
            
    \begin{quote}
Okay, the minimum release\_year is 2000.

Now, we will groupby the movies with their mean vote\_average by 2
variables: release\_year and genres, respectively:
\end{quote}

    \begin{Verbatim}[commandchars=\\\{\}]
{\color{incolor}In [{\color{incolor}22}]:} \PY{n}{df\PYZus{}genres\PYZus{}ops}\PY{o}{.}\PY{n}{groupby}\PY{p}{(}\PY{p}{[}\PY{l+s+s2}{\PYZdq{}}\PY{l+s+s2}{release\PYZus{}year}\PY{l+s+s2}{\PYZdq{}}\PY{p}{,} \PY{l+s+s2}{\PYZdq{}}\PY{l+s+s2}{genres}\PY{l+s+s2}{\PYZdq{}}\PY{p}{]}\PY{p}{)}\PY{p}{[}\PY{l+s+s2}{\PYZdq{}}\PY{l+s+s2}{vote\PYZus{}average}\PY{l+s+s2}{\PYZdq{}}\PY{p}{]}\PY{o}{.}\PY{n}{mean}\PY{p}{(}\PY{p}{)}
\end{Verbatim}


\begin{Verbatim}[commandchars=\\\{\}]
{\color{outcolor}Out[{\color{outcolor}22}]:} release\_year  genres         
         2000          action             5.957692
                       adventure          6.155556
                       animation          6.450000
                       comedy             6.034375
                       crime              6.206667
                       drama              6.585714
                       family             6.010000
                       fantasy            5.650000
                       history            6.250000
                       horror             5.587500
                       music              7.160000
                       mystery            6.216667
                       romance            5.984615
                       science fiction    5.453846
                       thriller           6.051724
                       war                6.200000
                       western            6.100000
         2001          action             5.975000
                       adventure          6.000000
                       animation          6.400000
                       comedy             5.917143
                       crime              6.150000
                       drama              6.535714
                       family             6.180000
                       fantasy            6.430000
                       history            6.433333
                       horror             5.812500
                       music              6.500000
                       mystery            6.340000
                       romance            6.180000
                                            {\ldots}   
         2014          drama              6.513208
                       family             6.625000
                       fantasy            6.560000
                       history            6.566667
                       horror             5.376190
                       music              6.685714
                       mystery            6.550000
                       romance            6.440741
                       science fiction    6.543478
                       thriller           6.177778
                       war                6.845455
                       western            6.200000
         2015          action             6.148780
                       adventure          6.326316
                       animation          6.469231
                       comedy             6.138710
                       crime              6.222222
                       documentary        6.575000
                       drama              6.484946
                       family             6.381250
                       fantasy            6.226667
                       history            6.733333
                       horror             5.528571
                       music              6.350000
                       mystery            6.100000
                       romance            6.440909
                       science fiction    6.271429
                       thriller           5.981356
                       war                6.633333
                       western            7.066667
         Name: vote\_average, Length: 285, dtype: float64
\end{Verbatim}
            
    \begin{quote}
This is what we needed.

Now there is a tricky part. I need to filter the max vote\_average mean
from each year group. To do that I need to convert this pandas.GroupBy
object to a dataframe first (Stackoverflow source):
\end{quote}

    \begin{Verbatim}[commandchars=\\\{\}]
{\color{incolor}In [{\color{incolor}23}]:} \PY{n}{df\PYZus{}grouped\PYZus{}mean} \PY{o}{=} \PY{n}{df\PYZus{}genres\PYZus{}ops}\PY{o}{.}\PY{n}{groupby}\PY{p}{(}\PY{p}{[}\PY{l+s+s2}{\PYZdq{}}\PY{l+s+s2}{release\PYZus{}year}\PY{l+s+s2}{\PYZdq{}}\PY{p}{,} \PY{l+s+s2}{\PYZdq{}}\PY{l+s+s2}{genres}\PY{l+s+s2}{\PYZdq{}}\PY{p}{]}\PY{p}{)}\PY{p}{[}\PY{l+s+s2}{\PYZdq{}}\PY{l+s+s2}{vote\PYZus{}average}\PY{l+s+s2}{\PYZdq{}}\PY{p}{]}\PY{o}{.}\PY{n}{mean}\PY{p}{(}\PY{p}{)}\PY{o}{.}\PY{n}{reset\PYZus{}index}\PY{p}{(}\PY{p}{)}
         \PY{n}{df\PYZus{}grouped\PYZus{}mean}\PY{o}{.}\PY{n}{head}\PY{p}{(}\PY{p}{)}
\end{Verbatim}


\begin{Verbatim}[commandchars=\\\{\}]
{\color{outcolor}Out[{\color{outcolor}23}]:}    release\_year     genres  vote\_average
         0          2000     action      5.957692
         1          2000  adventure      6.155556
         2          2000  animation      6.450000
         3          2000     comedy      6.034375
         4          2000      crime      6.206667
\end{Verbatim}
            
    \begin{quote}
In the next step, I re-groupby this new dataframe called
df\_grouped\_mean with its max vote\_average by release\_date column:
\end{quote}

    \begin{Verbatim}[commandchars=\\\{\}]
{\color{incolor}In [{\color{incolor}24}]:} \PY{n}{df\PYZus{}grouped\PYZus{}mean\PYZus{}max} \PY{o}{=} \PY{n}{df\PYZus{}grouped\PYZus{}mean}\PY{o}{.}\PY{n}{groupby}\PY{p}{(}\PY{l+s+s2}{\PYZdq{}}\PY{l+s+s2}{release\PYZus{}year}\PY{l+s+s2}{\PYZdq{}}\PY{p}{)}\PY{p}{[}\PY{l+s+s2}{\PYZdq{}}\PY{l+s+s2}{vote\PYZus{}average}\PY{l+s+s2}{\PYZdq{}}\PY{p}{]}\PY{o}{.}\PY{n}{max}\PY{p}{(}\PY{p}{)}\PY{o}{.}\PY{n}{reset\PYZus{}index}\PY{p}{(}\PY{p}{)}
         \PY{n}{df\PYZus{}grouped\PYZus{}mean\PYZus{}max}\PY{o}{.}\PY{n}{head}\PY{p}{(}\PY{p}{)}
\end{Verbatim}


\begin{Verbatim}[commandchars=\\\{\}]
{\color{outcolor}Out[{\color{outcolor}24}]:}    release\_year  vote\_average
         0          2000          7.16
         1          2001          6.94
         2          2002          7.20
         3          2003          7.45
         4          2004          6.55
\end{Verbatim}
            
    \begin{quote}
We have only last thing to do: Intersect (similar as inner join)
df\_grouped\_mean and df\_grouped\_mean\_max so that we get the genre
names of df\_grouped\_mean\_max entries (Stackoverflow source):
\end{quote}

    \begin{Verbatim}[commandchars=\\\{\}]
{\color{incolor}In [{\color{incolor}25}]:} \PY{n}{df\PYZus{}grouped\PYZus{}genres\PYZus{}vote\PYZus{}year} \PY{o}{=} \PY{n}{pd}\PY{o}{.}\PY{n}{merge}\PY{p}{(}\PY{n}{df\PYZus{}grouped\PYZus{}mean}\PY{p}{,} \PY{n}{df\PYZus{}grouped\PYZus{}mean\PYZus{}max}\PY{p}{,} \PY{n}{how}\PY{o}{=}\PY{l+s+s1}{\PYZsq{}}\PY{l+s+s1}{inner}\PY{l+s+s1}{\PYZsq{}}\PY{p}{,} \PY{n}{on}\PY{o}{=}\PY{p}{[}\PY{l+s+s2}{\PYZdq{}}\PY{l+s+s2}{release\PYZus{}year}\PY{l+s+s2}{\PYZdq{}}\PY{p}{,} \PY{l+s+s2}{\PYZdq{}}\PY{l+s+s2}{vote\PYZus{}average}\PY{l+s+s2}{\PYZdq{}}\PY{p}{]}\PY{p}{)}
         \PY{n}{df\PYZus{}grouped\PYZus{}genres\PYZus{}vote\PYZus{}year}
\end{Verbatim}


\begin{Verbatim}[commandchars=\\\{\}]
{\color{outcolor}Out[{\color{outcolor}25}]:}     release\_year       genres  vote\_average
         0           2000        music      7.160000
         1           2001          war      6.940000
         2           2002  documentary      7.200000
         3           2003  documentary      7.450000
         4           2004  documentary      6.550000
         5           2005      history      6.600000
         6           2006  documentary      6.600000
         7           2007  documentary      7.250000
         8           2008  documentary      7.600000
         9           2009  documentary      7.400000
         10          2010  documentary      7.360000
         11          2011  documentary      6.812500
         12          2012      western      7.700000
         13          2013  documentary      6.900000
         14          2014  documentary      7.800000
         15          2015      western      7.066667
\end{Verbatim}
            
    \begin{quote}
OK, thats it! We just need to plot this.
\end{quote}

    \begin{Verbatim}[commandchars=\\\{\}]
{\color{incolor}In [{\color{incolor}26}]:} \PY{n}{votes} \PY{o}{=} \PY{n}{df\PYZus{}grouped\PYZus{}genres\PYZus{}vote\PYZus{}year}\PY{o}{.}\PY{n}{groupby}\PY{p}{(}\PY{p}{[}\PY{l+s+s2}{\PYZdq{}}\PY{l+s+s2}{release\PYZus{}year}\PY{l+s+s2}{\PYZdq{}}\PY{p}{,} \PY{l+s+s2}{\PYZdq{}}\PY{l+s+s2}{genres}\PY{l+s+s2}{\PYZdq{}}\PY{p}{]}\PY{p}{)}\PY{p}{[}\PY{l+s+s1}{\PYZsq{}}\PY{l+s+s1}{vote\PYZus{}average}\PY{l+s+s1}{\PYZsq{}}\PY{p}{]}\PY{o}{.}\PY{n}{mean}\PY{p}{(}\PY{p}{)}
         \PY{n}{ax} \PY{o}{=} \PY{n}{votes}\PY{o}{.}\PY{n}{plot}\PY{p}{(}\PY{n}{kind}\PY{o}{=}\PY{l+s+s1}{\PYZsq{}}\PY{l+s+s1}{bar}\PY{l+s+s1}{\PYZsq{}}\PY{p}{,} \PY{n}{figsize}\PY{o}{=}\PY{p}{(}\PY{l+m+mi}{25}\PY{p}{,}\PY{l+m+mi}{7}\PY{p}{)}\PY{p}{)}\PY{p}{;}
         
         \PY{n}{ax}\PY{o}{.}\PY{n}{set\PYZus{}ylabel}\PY{p}{(}\PY{l+s+s2}{\PYZdq{}}\PY{l+s+s2}{Vote Averages}\PY{l+s+s2}{\PYZdq{}}\PY{p}{)}
         \PY{n}{ax}\PY{o}{.}\PY{n}{set\PYZus{}xlabel}\PY{p}{(}\PY{l+s+s2}{\PYZdq{}}\PY{l+s+s2}{Years, Genres}\PY{l+s+s2}{\PYZdq{}}\PY{p}{)}
         \PY{n}{ax}\PY{o}{.}\PY{n}{set\PYZus{}title}\PY{p}{(}\PY{l+s+s1}{\PYZsq{}}\PY{l+s+s1}{Highest Voted Movie Genres in 2000s}\PY{l+s+s1}{\PYZsq{}}\PY{p}{)}\PY{p}{;}
\end{Verbatim}


    \begin{center}
    \adjustimage{max size={0.9\linewidth}{0.9\paperheight}}{output_46_0.png}
    \end{center}
    { \hspace*{\fill} \\}
    
    \begin{quote}
We can see that "documentary" genre leaded most of the years in 2000s,
in terms of the mean vote average.
\end{quote}

    \subparagraph{2. Is there any relation between the movies' vote average
and revenue in terms of 2010
dollars?}\label{is-there-any-relation-between-the-movies-vote-average-and-revenue-in-terms-of-2010-dollars}

    \begin{quote}
To answer this question, we need to group revenue\_adj values to certain
thresholds and see the trends. Lets see some details.
\end{quote}

    \begin{Verbatim}[commandchars=\\\{\}]
{\color{incolor}In [{\color{incolor}28}]:} \PY{n}{df}\PY{o}{.}\PY{n}{describe}\PY{p}{(}\PY{p}{)}
\end{Verbatim}


\begin{Verbatim}[commandchars=\\\{\}]
{\color{outcolor}Out[{\color{outcolor}28}]:}                   id   vote\_count  vote\_average  release\_year   revenue\_adj
         count    3688.000000  3688.000000   3688.000000   3688.000000  3.688000e+03
         mean    47159.190618   565.996204      6.240076   2002.292842  1.440534e+08
         std     76316.028904   885.948174      0.765431     11.069910  2.187830e+08
         min         5.000000    45.000000      3.300000   1960.000000  2.370705e+00
         25\%      5250.250000   109.000000      5.700000   1996.000000  2.255749e+07
         50\%     11040.500000   240.500000      6.300000   2005.000000  6.975854e+07
         75\%     50327.750000   606.000000      6.800000   2011.000000  1.732400e+08
         max    417859.000000  9767.000000      8.400000   2015.000000  2.827124e+09
\end{Verbatim}
            
    \begin{quote}
It's a good idea to split revenue\_adj to 4 groups: 25\% (very low),
50\% (low), 75\% (normal) and 100\% (high), respectively (Udacity
source):
\end{quote}

    \begin{Verbatim}[commandchars=\\\{\}]
{\color{incolor}In [{\color{incolor}29}]:} \PY{c+c1}{\PYZsh{} split by percentiles}
         \PY{n}{percentiles} \PY{o}{=} \PY{n}{df}\PY{o}{.}\PY{n}{describe}\PY{p}{(}\PY{p}{)}\PY{o}{.}\PY{n}{loc}\PY{p}{[}\PY{p}{:}\PY{p}{,} \PY{l+s+s1}{\PYZsq{}}\PY{l+s+s1}{revenue\PYZus{}adj}\PY{l+s+s1}{\PYZsq{}}\PY{p}{]}\PY{o}{.}\PY{n}{tail}\PY{p}{(}\PY{l+m+mi}{4}\PY{p}{)}
         \PY{n}{first} \PY{o}{=} \PY{n}{df}\PY{o}{.}\PY{n}{query}\PY{p}{(}\PY{l+s+s1}{\PYZsq{}}\PY{l+s+s1}{revenue\PYZus{}adj \PYZlt{} }\PY{l+s+si}{\PYZob{}\PYZcb{}}\PY{l+s+s1}{\PYZsq{}}\PY{o}{.}\PY{n}{format}\PY{p}{(}\PY{n}{percentiles}\PY{p}{[}\PY{l+m+mi}{0}\PY{p}{]}\PY{p}{)}\PY{p}{)}
         \PY{n}{second} \PY{o}{=} \PY{n}{df}\PY{o}{.}\PY{n}{query}\PY{p}{(}\PY{l+s+s1}{\PYZsq{}}\PY{l+s+s1}{revenue\PYZus{}adj \PYZlt{} }\PY{l+s+si}{\PYZob{}\PYZcb{}}\PY{l+s+s1}{ \PYZam{} revenue\PYZus{}adj \PYZgt{}= }\PY{l+s+si}{\PYZob{}\PYZcb{}}\PY{l+s+s1}{\PYZsq{}}\PY{o}{.}\PY{n}{format}\PY{p}{(}\PY{n}{percentiles}\PY{p}{[}\PY{l+m+mi}{1}\PY{p}{]}\PY{p}{,} \PY{n}{percentiles}\PY{p}{[}\PY{l+m+mi}{0}\PY{p}{]}\PY{p}{)}\PY{p}{)}
         \PY{n}{third} \PY{o}{=} \PY{n}{df}\PY{o}{.}\PY{n}{query}\PY{p}{(}\PY{l+s+s1}{\PYZsq{}}\PY{l+s+s1}{revenue\PYZus{}adj \PYZlt{} }\PY{l+s+si}{\PYZob{}\PYZcb{}}\PY{l+s+s1}{ \PYZam{} revenue\PYZus{}adj \PYZgt{}= }\PY{l+s+si}{\PYZob{}\PYZcb{}}\PY{l+s+s1}{\PYZsq{}}\PY{o}{.}\PY{n}{format}\PY{p}{(}\PY{n}{percentiles}\PY{p}{[}\PY{l+m+mi}{2}\PY{p}{]}\PY{p}{,} \PY{n}{percentiles}\PY{p}{[}\PY{l+m+mi}{1}\PY{p}{]}\PY{p}{)}\PY{p}{)}
         \PY{n}{fourth} \PY{o}{=} \PY{n}{df}\PY{o}{.}\PY{n}{query}\PY{p}{(}\PY{l+s+s1}{\PYZsq{}}\PY{l+s+s1}{revenue\PYZus{}adj \PYZgt{}= }\PY{l+s+si}{\PYZob{}\PYZcb{}}\PY{l+s+s1}{\PYZsq{}}\PY{o}{.}\PY{n}{format}\PY{p}{(}\PY{n}{percentiles}\PY{p}{[}\PY{l+m+mi}{2}\PY{p}{]}\PY{p}{)}\PY{p}{)}
         
         \PY{n}{mean\PYZus{}quality\PYZus{}first} \PY{o}{=} \PY{n}{first}\PY{p}{[}\PY{l+s+s1}{\PYZsq{}}\PY{l+s+s1}{vote\PYZus{}average}\PY{l+s+s1}{\PYZsq{}}\PY{p}{]}\PY{o}{.}\PY{n}{mean}\PY{p}{(}\PY{p}{)}
         \PY{n}{mean\PYZus{}quality\PYZus{}second} \PY{o}{=} \PY{n}{second}\PY{p}{[}\PY{l+s+s1}{\PYZsq{}}\PY{l+s+s1}{vote\PYZus{}average}\PY{l+s+s1}{\PYZsq{}}\PY{p}{]}\PY{o}{.}\PY{n}{mean}\PY{p}{(}\PY{p}{)}
         \PY{n}{mean\PYZus{}quality\PYZus{}third} \PY{o}{=} \PY{n}{third}\PY{p}{[}\PY{l+s+s1}{\PYZsq{}}\PY{l+s+s1}{vote\PYZus{}average}\PY{l+s+s1}{\PYZsq{}}\PY{p}{]}\PY{o}{.}\PY{n}{mean}\PY{p}{(}\PY{p}{)}
         \PY{n}{mean\PYZus{}quality\PYZus{}fourth} \PY{o}{=} \PY{n}{fourth}\PY{p}{[}\PY{l+s+s1}{\PYZsq{}}\PY{l+s+s1}{vote\PYZus{}average}\PY{l+s+s1}{\PYZsq{}}\PY{p}{]}\PY{o}{.}\PY{n}{mean}\PY{p}{(}\PY{p}{)}
         \PY{n+nb}{print}\PY{p}{(}\PY{n}{mean\PYZus{}quality\PYZus{}first}\PY{p}{)}
         \PY{n+nb}{print}\PY{p}{(}\PY{n}{mean\PYZus{}quality\PYZus{}second}\PY{p}{)}
         \PY{n+nb}{print}\PY{p}{(}\PY{n}{mean\PYZus{}quality\PYZus{}third}\PY{p}{)}
         \PY{n+nb}{print}\PY{p}{(}\PY{n}{mean\PYZus{}quality\PYZus{}fourth}\PY{p}{)}
\end{Verbatim}


    \begin{Verbatim}[commandchars=\\\{\}]
6.152711496746197
6.113015184381772
6.229284164859011
6.465292841648589

    \end{Verbatim}

    \begin{quote}
We can now plot a chart to see the trend.
\end{quote}

    \begin{Verbatim}[commandchars=\\\{\}]
{\color{incolor}In [{\color{incolor}30}]:} \PY{c+c1}{\PYZsh{} Create a bar chart with proper labels}
         \PY{n}{locations} \PY{o}{=} \PY{p}{[}\PY{l+m+mi}{1}\PY{p}{,} \PY{l+m+mi}{2}\PY{p}{,} \PY{l+m+mi}{3}\PY{p}{,} \PY{l+m+mi}{4}\PY{p}{]}
         \PY{n}{heights} \PY{o}{=} \PY{p}{[}\PY{n}{mean\PYZus{}quality\PYZus{}first}\PY{p}{,} \PY{n}{mean\PYZus{}quality\PYZus{}second}\PY{p}{,} \PY{n}{mean\PYZus{}quality\PYZus{}third}\PY{p}{,} \PY{n}{mean\PYZus{}quality\PYZus{}fourth}\PY{p}{]}
         \PY{n}{labels} \PY{o}{=} \PY{p}{[}\PY{l+s+s1}{\PYZsq{}}\PY{l+s+s1}{Very Low}\PY{l+s+s1}{\PYZsq{}}\PY{p}{,} \PY{l+s+s1}{\PYZsq{}}\PY{l+s+s1}{Low}\PY{l+s+s1}{\PYZsq{}}\PY{p}{,} \PY{l+s+s1}{\PYZsq{}}\PY{l+s+s1}{Normal}\PY{l+s+s1}{\PYZsq{}}\PY{p}{,} \PY{l+s+s1}{\PYZsq{}}\PY{l+s+s1}{High}\PY{l+s+s1}{\PYZsq{}}\PY{p}{]}
         \PY{n}{plt}\PY{o}{.}\PY{n}{bar}\PY{p}{(}\PY{n}{locations}\PY{p}{,} \PY{n}{heights}\PY{p}{,} \PY{n}{tick\PYZus{}label}\PY{o}{=}\PY{n}{labels}\PY{p}{)}
         \PY{n}{plt}\PY{o}{.}\PY{n}{title}\PY{p}{(}\PY{l+s+s1}{\PYZsq{}}\PY{l+s+s1}{Movie Vote Averages by Revenue in Terms of 2010 Dollars}\PY{l+s+s1}{\PYZsq{}}\PY{p}{)}
         \PY{n}{plt}\PY{o}{.}\PY{n}{xlabel}\PY{p}{(}\PY{l+s+s1}{\PYZsq{}}\PY{l+s+s1}{Revenues (in terms of 2010 dollars, accounting for inflation over time)}\PY{l+s+s1}{\PYZsq{}}\PY{p}{)}
         \PY{n}{plt}\PY{o}{.}\PY{n}{ylabel}\PY{p}{(}\PY{l+s+s1}{\PYZsq{}}\PY{l+s+s1}{Movie Vote Average}\PY{l+s+s1}{\PYZsq{}}\PY{p}{)}\PY{p}{;}
\end{Verbatim}


    \begin{center}
    \adjustimage{max size={0.9\linewidth}{0.9\paperheight}}{output_54_0.png}
    \end{center}
    { \hspace*{\fill} \\}
    
     \#\# Conclusions

\subparagraph{1. Which genres has the highest average of votes from year
to year in
2000s?}\label{which-genres-has-the-highest-average-of-votes-from-year-to-year-in-2000s}

\begin{quote}
We can see that in 2000s, "documentary" genre has the most vote\_average
count record by year to year (2002, 2003, 2004, 2006, 2007, 2008, 2009,
2010, 2011, 2013, 2014). We can say that people vote higher points to
documentary movies, when compared to movies with other genres. Another
interesting point is that "music" genre came into prominence in year
2000, which has mean vote\_average of 7.16.
\end{quote}

\subparagraph{2. Is there any relation between the movies' vote average
and revenue in terms of 2010
dollars?}\label{is-there-any-relation-between-the-movies-vote-average-and-revenue-in-terms-of-2010-dollars}

\begin{quote}
I tried to find a correlation between movie revenues in terms of 2010
dollars and its vote\_average. Before making this exploration, my
opinion was higher vote averages means higher revenues. I can say that I
was right, because the highest vote\_average mean belongs to the highest
revenue percentile (after 75\%), while the second one belongs to the
second hisghest revenue percentile (after 50\%).
\end{quote}


    % Add a bibliography block to the postdoc
    
    
    
    \end{document}
